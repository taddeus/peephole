\documentclass[10pt,a4paper]{article}
\usepackage[latin1]{inputenc}
\usepackage{amsmath}
\usepackage{amsfonts}
\usepackage{amssymb}
\usepackage{booktabs}
\usepackage{graphicx}
\usepackage{listings}
\usepackage{subfigure}
\usepackage{float}
\usepackage{hyperref}

\title{Peephole Optimizer}
\author{Jayke Meijer (6049885), Richard Torenvliet (6138861), Taddeus Kroes (6054129)}

\begin{document}
\maketitle

\section{Introduction}
The goal of the assignment is to implement the optimization stage of the compiler. To reach this goal the parser part of the compiler has to be implemented.

The output of the gcc cross compiler on a c program is our input, the output of the gcc cross compiler is in the form of Assembly code, but not optimized. Our assignment includes a number of c programs, an important part of the assignment is parsing the data. Parsing the data is done with lex and yacc. The lexer is a program that finds keywords that meets the regular expression provided in the lexer. After the lexer, the yaccer takes over. Yaccer can turn the keywords in to an action.

\section{Design \& Implementation}
We decided to implement the optimization in python. We chose this programming language because python is an easy language to manipulate strings, work objective ori\"ented etc.
It turns out that a lex and yacc are also implemented in a python version, named PLY(Python Lex-Yacc). This allows us to use one language, Python, instead of two i.e. C and Python. Also no debugging is needed in C, only in Python which makes our assignment more feasible.

\subsection{Design}


\subsection*{Implementation}
This 

\subsubsection*{PLY}

\section{Results}

\subsection*{pi.c}

\subsection*{arcron.c}

\subsection*{whet.c}

\subsection*{slalom.c}

\subsection*{clinpack.c}

\section{conclusion}
\end{document}