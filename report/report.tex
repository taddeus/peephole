\documentclass[10pt,a4paper]{article}
\usepackage[latin1]{inputenc}
\usepackage{amsmath,amsfonts,amssymb,booktabs,graphicx,listings,subfigure}
\usepackage{float,hyperref}

% Paragraph indentation
\setlength{\parindent}{0pt}
\setlength{\parskip}{1ex plus 0.5ex minus 0.2ex}

\title{Peephole Optimizer}
\author{Jayke Meijer (6049885), Richard Torenvliet (6138861), Tadde\"us Kroes
    (6054129)}

\begin{document}

\maketitle
\tableofcontents

\pagebreak

\section{Introduction}

The goal of the assignment is to implement the peephole optimization stage of
xgcc cross compiler. This requires a MIPS Assembly parser to parse the output
of the compiler. Also, an assembly writer is needed to write the optimized
statements back to valid Assembly code for the assembler.

The assignment provides a number of benchmarks written in C. The objective is
to obtain a high speedup in number of cycles for these benchmarks.

\section{Types of optimizations}

There are two general types of optimizations on the assembly code: global
optimizations and optimizations on so-called basic blocks. These optimizations
will be discussed individually below.

\subsection{Global optimizations}

We only perform one global optimization, which is optimizing branch-jump
statements. The unoptimized Assembly code contains sequences of statements with
the following structure:
\begin{verbatim}
    beq ...,$Lx
    j $Ly
$Lx:   ...
\end{verbatim}
%This is inefficient, since there is a branch to a label that follows this code.
In this code, it is more efficient to replace the branch statement with a
\texttt{bne} (the opposite case) to the label used in the jump statement. This
way, the jump statement can be eliminated since the label directly follows it.
The same can be done for the opposite case, where a \texttt{bne} is changed
into a \texttt{beq}.

Since this optimization is done between two series of codes with jumps and
labels, we can not perform this code during the basic block optimizations.

\subsection{Basic Block Optimizations}

Optimizations on basic blocks are a more extended part of the optimizer.

First of all, what is a basic block? A basic block is a sequence of statements
guaranteed to be executed in that order, and that order alone. This is the case
for a piece of code not containing any branches or jumps (except for the last
statement).

To divide the code into basic blocks, the ``leaders'' have to be found. A
leading statement is a leader if it is either a jump or branch statement, or
the target of such a statement. Each leader is the start of a new basic block.

There are five types of optimizations performed on basic blocks in our
implementation. Each is described individually below.

\subsubsection{Standard peephole optimizations}

These are optimizations that look for a certain statement or pattern of
statements, and optimize these. For example,
\begin{verbatim}
mov $regA,$regB
instr $regA, $regA,...
\end{verbatim}
can be optimized to:
\begin{verbatim}
instr $regA, $regB,...
\end{verbatim}
\texttt{\$regA} should contain the same value as \texttt{\$regB} after the move
statement, so \texttt{\$regB} can be used by \texttt{instr}.  Since
\texttt{instr} overwrites \texttt{\$regA}, the move statement has not further
effect after \texttt{instr} and can be removed.

There are a few more of these cases, which are described on the practicum page
\footnote{\url{http://staff.science.uva.nl/~andy/compiler/prac.html}} and in
Appendix \ref{opt}.

\subsubsection{Common subexpression elimination}

A more advanced optimization is common subexpression elimination. This means
that expensive operations like multiplications or additions are performed only
once and the result is then `copied' into registers where needed.
\begin{verbatim}
addu $2,$4,$3              addu = $8, $4, $3  # $8 is free
...                        mov = $2, $8
...                   ->   ...
...                        ...
addu $5,$4,$3              mov = $4, $8
\end{verbatim}

A standard method for doing this is usage of a DAG or Directed Acyclic Graph.
However, this requires either the code to be in Static single
assignment
form\footnote{\url{http://en.wikipedia.org/wiki/Static\_single\_assignment\_form}},
or an advanced liveness check. Our implementation contains a (partially tested)
implementation of DAG creation, but this is not used in the final
implementation.  However, our implementation does contain a simplified version
of common subexpression elimination:

The statement list of a block is traversed in reversed order, looking for
instructions that are eligible for CSE (\texttt{addu}, for example). If such an
instruction is found, it is marked and the rest of the statement list is
traversed while marking all statements that are equal to the found instruction.
If a statement assigns a register that is uses by the instruction, traversal
stops.

If more than one instruction have been marked, a new instruction is inserted
above the first occurrence (the last occurrence in reversed order). This
instruction performs the calculation and saves it in a free temporary register.
Then, each occurrence is replaced by a \texttt{move} of the free register to
its original destination register.

This method is obviously less efficient method then the DAG.  However, since
the basic blocks are generally not very large and the execution time of the
optimizer is not a primary concern, this is not a large problem.

\subsubsection{Constant folding}

Constant folding is an optimization where the outcome of arithmetics are
calculated at compile time. If a register x is known to contain a constant
value, all following uses of \texttt{x} can be replaced by that value until a
redefinition of x.

Arithmetics in Assembly are always performed between two registers or a
register and a constant. If the current value of all used registers is known,
The expression can be executed at-compile-time and the instruction can be
replaced by an immediate load of the result. See \ref{opt} for an example.

%In other words until the current definition of \texttt{x} becomes dead.
%Therefore reaching definitions analysis is needed. Reaching definitions is a
%form of liveness analysis, we use the liveness analysis within a block and not
%between blocks.

During the constant folding, so-called algebraic transformations are performed
as well. When calculations are performed using constants, some calculations can
be replaced by a load- or move-instruction. An example is the statement
$x = y + 0$, or in Assembly: \texttt{addu \$1, \$2, 0}. This can be replaced by
$x = y$ or \texttt{move \$1, \$2}. A list of transformations that are performed
can be found in appendix \ref{opt}.

\subsubsection{Copy propagation}

Copy propagation replaces usage of registers that have been assigned the value
of another register earlier. In Assembly code, such an assignment is in the
form of a \texttt{move} instruction.

This is not a direct optimization, but is often does create dead code (the
\texttt{move} statement) that can be eliminated.

To perform copy propagation within the same basic block, the block is traversed
until a \texttt{move x, y} instruction is encountered. For each of these ``copy
statements'', the rest of the block is traversed while looking for usage of the
\texttt{move}'s destination address \texttt{x}. These usages are replaced by
usages of \texttt{y}, until either \texttt{x} or \texttt{y} is re-assigned.

%Copy propagation `unpacks' a move instruction, by replacing its destination
%address with its source address in the code following the move instruction.
%
%This is not a direct optimization, but this does allow for a more effective
%dead code elimination.
%
%The code of the block is traversed linearly. If a move operation is
%encountered, the source and destination address of this move are stored. If a
%normal operation with a source and a destination address are found, a number of
%checks are performed.
%
%The first check is whether the destination address is stored as a destination
%address of a move instruction. If so, this move instruction is no longer valid,
%so the optimizations can not be done. Otherwise, continue with the second
%check.
%
%In the second check, the source address is compared to the destination
%addresses of all still valid move operations. If these are the same, in the
%current operation the found source address is replaced with the source address
%of the move operation.

An example would be the following:
\begin{verbatim}
move $regA, $regB                    move $regA, $regB
...                                  ...
Code not writing $regA or $regB  ->  ...
...                                  ...
addu $regC, $regA, ...               addu $regC, $regB, ...
\end{verbatim}
\texttt{\$regA} is replaced with \texttt{\$regB}. Now, the move instruction
might have become useless. If so, it will be removed by dead code elimination.

To also replace usages in successors of the basic block, a Reaching Definitions
analysis is used: If a \texttt{move}-statement is in the $REACH_{out}$ set of
the block, it is used in one of the block's successors. To be able to replace a
usage, the definition must me the only definition reaching the usage. To
determine this, copy propagation defines a new dataflow problem that yields the
$COPY_{in}$ and $COPY_{out}$ sets. the successor The definition is the only
reaching definition if it is in the successor's $COPY_{in}$ set. If this is the
case, the usage van be replaced by the destination address of the
\texttt{move}-statement. \\
Note: Though we implemented the algorithm as described above, we did not
encounter any replacements between basic blocks while optimizing the provided
benchmark scripts. This might mean that our implementation of the copy
propagation dataflow problem is based on the lecture slides, which only briefly
describe the algorithm.

\subsubsection{Dead code elimination}

The final optimization that is performed is dead code elimination. This removes
statements of which the result is never used.

To determine if a register is used from a certain point in the code, liveness
analysis is used. A variable is ``live'' at a certain point in the code if it
holds a value that may be needed in the future. Using the $LIVE_{out}$ set
that is generated by the analysis, we can check if a register is dead after a
certain point in a basic block. Each statement that assigns a register which
is dead from that point on is removed.

\section{Implementation}

We decided to implement the optimizations in Python. We chose this programming
language because Python is an easy language to manipulate strings, work
object-oriented etc..

To implement the parser, we use a Python variant of Yacc and Lex named
PLY(Python Lex-Yacc). By using this module instead of the regular C
implementations of Yacc and Lex, we only use a single language in the entire
project.

The program has three steps:
\begin{enumerate}
    \item Parsing the Assembly code to an Intermediate Representation (IR).
    \item Performing optimizations on the IR.
    \item Writing the IR back to Assembly code.
\end{enumerate}

Our code is provided with this report, and is also available on GitHub: \\
\url{https://github.com/taddeus/peephole}

\subsection{Structure}

% TODO

\subsection{Parsing}

The parser is implemented using PLY, which uses standard Lex-Yacc syntax in
given function formats.

The parser assumes that it is given valid Assembly code as input, so it does
not validate whether, for example, command arguments are valid. This design
decision was made because the optimizer uses the output of a compiler, which
should produce valid Assembly code.

The parser recognizes 4 types of ``statements'':
\begin{itemize}
    \item \textbf{comment} Line starting with a `\#'.
    \item \textbf{directive} C-directive, used by the compiler. These are
                             matched and treated in the same way as comments.
    \item \textbf{command} Machine instruction, followed 0 to 3 arguments and
                           optionally an inline comment.
    \item \textbf{label} Line containing a \texttt{WORD} token, followed by a
                         colon (`:').
\end{itemize}

Each statement is represented by a \texttt{Statement} object containing a type,
a name, optionally a list of arguments and optionally a list of extra options
(such as inline comments). The parsed list of statements forms a
\texttt{Program} object, which is the return value of the parser.

\subsection{Optimization loop}

The optimizations are performed in a loop until no more changed are made. The
optimization loop first performs global optimizations on the entire statement
list of the program. Second, all dataflow analyses are performed (basic block
creation, flow graph generation, liveness, reaching definitions, copy
propagation). Finally, all basic block-level optimizations are executed. if
either the global or one of the block optimizations yields a change in
statements, another iteration is executed.

\subsection{Writing}

Once all the optimizations have been done, the IR needs to be rewritten to
Assembly code. After this step, the xgcc cross compiler can make binary code
from the generated Assembly code.

The writer expects a list of statements, so first the blocks have to be
concatenated again into a list. After this is done, the list is passed on to
the writer, which writes the instructions back to Assembly and saves the file.
We believe that the writer code is self-explanatory, so we will not discuss it
in detail here.

The writer has a slightly different output format than the xgcc compiler in
some cases. Therefore, the main execution file has an option to also write the
original statement list back to a files way, differences in tabs, spaces and
newlines do not show up when checking the differences between optimized and
non-optimized files.

\subsection{Execution}

To execute the optimizer, the following command can be given: \\
\texttt{./main.py <original file> <optimized file> <rewritten original file>} \\
There is also a script available that runs the optimizer and automatically
starts the program \emph{meld}. In meld it is easy to visually compare the
original file and the optimized file. The command to execute this script is: \\
\texttt{./run <benchmark name (e.g. whet)>}

\section{Testing}

Of course, it has to be guaranteed that the optimized code still functions
exactly the same as the none-optimized code. To do this, testing is an
important part of out program. We have two stages of testing. The first stage
is unit testing. The second stage is to test whether the compiled code has
exactly the same output.

\subsection{Unit testing}

For almost every piece of important code, unit tests are available. Unit tests
give the possibility to check whether each small part of the program, for
instance each small function, is performing as expected. This way bugs are
found early and very exactly. Otherwise, one would only see that there is a
mistake in the program, not knowing where this bug is. Naturally, this means
debugging is a lot easier.

The unit tests can be run by executing \texttt{make test} in the root folder of
the project. This does require the \texttt{testrunner} module of Python.

Also available is a coverage report. This report shows how much of the code has
been unit tested. To make this report, the command \texttt{make coverage} can
be run in the root folder. The report is than added as a folder \emph{coverage}
in which a \emph{index.html} can be used to see the entire report.

\subsection{Ouput comparison}

In order to check whether the optimization does not change the functioning of
the program, the output of the provided benchmark programs has to be compared
to the output after optimization. If any of these outputs is not equal to the
original output, our optimizations are to aggressive, or there is a bug
somewhere in the code.

\section{Results}

The following results have been obtained:\\
\begin{tabular}{|c|c|c|c|c|c|}
\hline
Benchmark & Original     & Removed      & Original & Optimized & Performance \\
          & Instructions & instructions & cycles   & cycles    & boost(cycles) \\
\hline
pi        &           94 &            2 &          &           &             \% \\
acron     &          361 &           24 &          &           &             \% \\
dhrystone &          752 &           52 &          &           &             \% \\
whet      &          935 &           37 &          &           &             \% \\
slalom    &         4177 &          227 &          &           &             \% \\
clinpack  &         3523 &              &          &           &              \% \\
\hline
\end{tabular}

\begin{tabular}{|c|c|c|c|c|c|}
\hline
Benchmark & Original     & Removed      & Original & Optimized & Performance \\
          & Instructions & instructions & cycles   & cycles    & boost(cycles)\\
\hline
pi        &           94 &      2       &    1714468   &   1714362      &   0.006182676 \%       \\
acron     &          361 &      19      &    4435687   &   4372825      &   1.417187462 \%       \\
dhrystone &          752 &      36      &    2887710   &   2742720      &   5.020933542 \%       \\
whet      &          935 &      23      &    2864526   &   2840042      &   0.854731289 \%       \\
slalom    &         4177 &      107     &    2879140   &   2876105      &   0.143480345 \%       \\
clinpack  &         3523 &      49      &    1543746   &   1528406      &   1.353201887  \%       \\
\hline
\end{tabular}

\pagebreak
\appendix

\section{List of all optimizations}

\label{opt}

\textbf{Global optimizations}

\begin{verbatim}
    beq ...,$Lx             bne ...,$Ly
    j $Ly               ->  $Lx:   ...
$Lx:   ...


    bne ...,$Lx             beq ...,$Ly
    j $Ly               ->  $Lx:   ...
$Lx:   ...
\end{verbatim}
\textbf{Standard basic block optimizations}

\begin{verbatim}
mov $regA,$regA         ->  --- // remove it


mov $regA,$regB         ->  instr $regA, $regB,...
instr $regA, $regA,...


instr $regA,...             instr $4,...
mov [$4-$7], $regA      ->  jal XXX
jal  XXX


sw $regA,XXX            ->  sw $regA, XXX
ld $regA,XXX


shift $regA,$regA,0     ->  --- // remove it


add $regA,$regA,X       ->  lw ...,X($regA)
lw ...,0($regA)
\end{verbatim}
\textbf{Advanced basic block optimizations}

\begin{verbatim}
# Common subexpression elimination
addu $regA, $regB, 4        addu $regD, $regB, 4
...                         move $regA, $regD
Code not writing $regB  ->  ...
...                         ...
addu $regC, $regB, 4        move $regC, $regD

# Constant folding
li $2, 2                    $2 = 2
sw $2, 16($fp)              16($fp) = 2
li $2, 3                    $2 = 3
sw $2, 20($fp)          ->  20($fp) = 3
lw $2, 16($fp)              $2 = 16($fp) = 2
lw $3, 20($fp)              $3 = 20($fp) = 3
addu $2, $2, $3             change to "li $2, 0x00000005"

# Copy propagation
move $regA, $regB           move $regA, $regB
...                         ...
Code not writing $regA, ->  ...
$regB                       ...
...                         ...
addu $regC, $regA, ...      addu $regC, $regB, ...


# Algebraic transformations
addu $regA, $regB, 0    ->  move $regA, $regB

subu $regA, $regB, 0    ->  move $regA, $regB

mult $regA, $regB, 1    ->  move $regA, $regB

mult $regA, $regB, 0    ->  li   $regA, 0

mult $regA, $regB, 2    ->  sll  $regA, $regB, 1
\end{verbatim}

\end{document}
